\documentclass{sasreport}

%%
%% User settings
%%

\classno{}
\stuno{}
\groupno{}
\stuname{}
\expdate{\expdatefmt\today}
\expidx{一}
\expname{信号的特性}

%%
%% Document body
%%

\begin{document}
% First page
% Some titles and personal information are defined in ``\maketitle''.
\maketitle

\section{实验前思考题}
% See: https://tex.stackexchange.com/a/229542
\begin{enumerate}[(1)]
    \item 学习和掌握ELVIS II+虚拟仪器平台和Emona SIGEx实验板的使用方法;
    \item 了解基带低通滤波器、可调谐低通滤波器、RC电路对信号的影响;
    \item 了解限幅器的规律;
    \item 掌握脉冲序列速率受系统特性影响的规律,及计算最大转换速率方法。
\end{enumerate}

\section{实验记录}
\begin{table}[H]
    \centering
    \caption{\bfseries 两种滤波器的转换时间}
    \large
    \begin{tabularx}{\textwidth}{|c|Y|Y|Y|Y|}
        \hline
        测量范围 (\%) & BLPF@1kHz ($\mu$s) & TLPF@1kHz ($\mu$s) & BLPF@1.5kHz ($\mu$s) & TLPF@1.5kHz ($\mu$s) \\\hline
        上升 10-90  &                    &                    &                      &                      \\\hline
        下降 10-90  &                    &                    &                      &                      \\\hline
    \end{tabularx}
\end{table}

\begin{center}
    图1.9(用自己实测数据绘制)
\end{center}

\begin{table}[H]
    \centering
    \caption{\bfseries 三种滤波器的转换时间}
    \large
    \begin{tabularx}{\textwidth}{|c|Y|Y|Y|}
        \hline
        测量范围 (\%) & BLPF@250Hz ($\mu$s) & TLPF@250Hz ($\mu$s) & RCLPF@250Hz ($\mu$s) \\\hline
        上升 10-90  &                     &                     &                      \\\hline
        下降 10-90  &                     &                     &                      \\\hline
    \end{tabularx}
\end{table}

计算三种滤波器每秒能够转换脉冲波的最大次数:

\noindent
\begin{tabularx}{\textwidth}{cYcYcY}
    BLPF: &  & ,TLPF: &  & ,RCLPF: & \\\Xcline{2-2}{1pt}\Xcline{4-4}{1pt}\Xcline{6-6}{1pt}
\end{tabularx}

\begin{table}[H]
    \centering
    \caption{\bfseries 滤波器与正弦波频率的关系}
    \large
    \begin{tabularx}{.8\textwidth}{|c|Y|Y|Y|}
        \hline
        频率 (kHz) & BLPF ($V_\mathrm{pp}$) & TLPF ($V_\mathrm{pp}$) & RCLPF ($V_\mathrm{pp}$) \\\hline
        0.25     &                        &                        &                         \\\hline
        0.5      &                        &                        &                         \\\hline
        0.8      &                        &                        &                         \\\hline
        1        &                        &                        &                         \\\hline
        1.5      &                        &                        &                         \\\hline
        2        &                        &                        &                         \\\hline
        2.5      &                        &                        &                         \\\hline
        3        &                        &                        &                         \\\hline
        4        &                        &                        &                         \\\hline
        5        &                        &                        &                         \\\hline
        8        &                        &                        &                         \\\hline
        10       &                        &                        &                         \\\hline
    \end{tabularx}
\end{table}

\begin{center}
    图 1.12(用自己实测数据绘制)
\end{center}

\begin{center}
    限幅器输入输出电压的峰峰值关系图(选做)
\end{center}

\section{实验思考题}
问题 1:

两个相邻脉冲的最小间隔为 \underline{\ \ \ },最大间隔为 \underline{\ \ \ }.

问题 2:

信号在通过基带低通滤波器、可调谐低通滤波器时,发生的哪些变化?

信号在转换的过程中是否平缓?两种波形有哪些不同?

问题 3:

逐渐提高时钟频率时,BLPF和TLPF的输出波形有何改变?

频率大约达到  \underline{\ \ \ } Hz时,BLPF输出发生很剧烈的变化甚至失真。

频率大约达到  \underline{\ \ \ } Hz时,TLPF输出发生很剧烈的变化甚至失真。

问题 4:

根据1.5.3的测量数据,BLPF和TLPF可以正常工作的频率上限是多少?

问题 5 (选做):

通过信号发生器上的幅值(AMPLITUDE)控件改变信号的幅值,限幅器的输出会发生什么变化?

\section{实验过程与数据分析}

\section{实验体会与建议}

\end{document}
