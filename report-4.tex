\documentclass{sasreport}

%%
%% User settings
%%

\classno{}
\stuno{}
\groupno{}
\stuname{}
\expdate{\expdatefmt\today}
\expidx{四}
\expname{卷积(卷积和)}

%%
%% Document body
%%

\begin{document}
% First page
% Some titles and personal information are defined in ``\maketitle''.
\maketitle

\section{实验前思考题}
1、写出卷积积分的计算步骤和公式:

2、$f(t)$ 与冲激函数的卷积如何表示:

\section{实验记录}
\subsection{单位脉冲响应}
调节 $a_0$ 的增益值,结合示波器观察,使脉冲的幅值(即加法器 a 的输出)精确地达到1V,记录的值为 \underline{\ \ \ }。

\begin{figure}[H]
    \caption{延迟线路的输入信号(即第一个 $Z^{-1}$ 模块的输入)和加法器的输出信号}
    \centering

\end{figure}
\noindent 输出序列中各个脉冲的幅值分别是:\underline{\ \ \ }, \underline{\ \ \ }, \underline{\ \ \ }.

\subsection{输入脉冲对}
$h(0),h(1),h(2),h(3)$ 的幅值分别为: \underline{\ \ \ }, \underline{\ \ \ }, \underline{\ \ \ }, \underline{\ \ \ }.

验证输出序列是否仅为两个偏置单位脉冲响应的和,并说明你是如何进行验证的。要求:验证时,不能只通过图形观察,必须给出测量数据。

\subsection{正弦波整流输入}
\begin{figure}[H]
    \caption{采样保持器的输入和输出}
    \centering

\end{figure}

\begin{table}[H]
    \caption{系统叠加性验证}
    \centering
    \begin{tabularx}{\textwidth}{|c|Y|Y|Y|Y|Y|} \hline
        \multirow{2}{*}{\diagbox{序号}{幅值}} & \multirow{2}{*}{输入 (V)} & \multicolumn{4}{c|}{输出 (V)}         \\\Xcline{3-6}{1pt}
                                          &                         & \makecell{$b_0=0.3$                 \\ $b_1=0.5$ \\ $b_2=-0.2$}                           & \makecell{$b_0=0.3$ \\ $b_1=0$ \\ $b_2=0$} & \makecell{$b_0=0$ \\ $b_1=0.5$ \\ $b_2=0$} & \makecell{$b_0=0$ \\ $b_1=0$ \\ $b_2=-0.2$} \\\hline
        0                                 &                         &                             &  &  & \\\hline
        1                                 &                         &                             &  &  & \\\hline
        2                                 &                         &                             &  &  & \\\hline
        3                                 &                         &                             &  &  & \\\hline
        4                                 &                         &                             &  &  & \\\hline
        5                                 &                         &                             &  &  & \\\hline
        6                                 &                         &                             &  &  & \\\hline
        7                                 &                         &                             &  &  & \\\hline
        8                                 &                         &                             &  &  & \\\hline
    \end{tabularx}
\end{table}

\subsection{正弦输入}
\begin{figure}[H]
    \caption{系统输入(即S/H的输出)和系统输出(即加法器的输出Y)}
    \centering

\end{figure}

\begin{table}[H]
    \caption{正弦波作为输入的输出}
    \centering
    \begin{tabularx}{\textwidth}{|c|Y|Y|Y|Y|Y|} \hline
        \multirow{2}{*}{\diagbox{序号}{幅值}} & \multirow{2}{*}{输入 (V)} & \multicolumn{4}{c|}{输出 (V)}         \\\Xcline{3-6}{1pt}
                                          &                         & \makecell{$b_0=0.3$                 \\ $b_1=0.5$ \\ $b_2=-0.2$}                           & \makecell{$b_0=0.3$ \\ $b_1=0$ \\ $b_2=0$} & \makecell{$b_0=0$ \\ $b_1=0.5$ \\ $b_2=0$} & \makecell{$b_0=0$ \\ $b_1=0$ \\ $b_2=-0.2$} \\\hline
        0                                 &                         &                             &  &  & \\\hline
        1                                 &                         &                             &  &  & \\\hline
        2                                 &                         &                             &  &  & \\\hline
        3                                 &                         &                             &  &  & \\\hline
        4                                 &                         &                             &  &  & \\\hline
        5                                 &                         &                             &  &  & \\\hline
        6                                 &                         &                             &  &  & \\\hline
        7                                 &                         &                             &  &  & \\\hline
        8                                 &                         &                             &  &  & \\\hline
    \end{tabularx}
\end{table}

\subsection{特殊应用 (选做)}
\begin{figure}[H]
    \caption{分别记录第一个、第二个设置时的波形图(选做)}
    \centering

\end{figure}

\section{实验思考题}
\textbf{问题 1:}

“叠加性”指什么?根据实验结果,如何体现叠加性与可加性原理的。

\textbf{问题 2:}

若本实验拓展到更多的连续脉冲,你认为会看到什么现象?请加以说明。

\textbf{问题 3:}

测量并记录正弦波经过半波整流后的幅值,并解释其幅值减少是为什么。

\textbf{问题 4:}

脉冲发生器设置为其他频率时,采样会发生什么变化?是否频率越大越好,为什么?

\textbf{问题 5:}

这个过程与叠加原理有何关系?(可以用数学的方式来表示实验现象)

\textbf{问题 6:}

写出 $y_2$ 和 $y_1$ 的表达式,讨论它们有何不同点。

\textbf{问题 7 (选做):}

列出平方和计算的结果,并计算标准差和均值。

\textbf{问题 8 (选做):}

适当地改变正弦信号的频率,当正弦输入的频率在100Hz附近变化时,其振幅将会发生什么变化?

\section{实验过程与数据分析}
 {\kaishu (可以写实验中遇到的问题及解决方式,以及叙述具体实验过程,记录实验数据在原始数据表格,如需要引用原始数据表格,请标注出表头,如“实验记录见表2-*”)}

\section{实验体会与建议}

\end{document}
