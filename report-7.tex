\documentclass{sasreport}

%%
%% User settings
%%

\classno{}
\stuno{}
\groupno{}
\stuname{}
\expdate{\expdatefmt\today}
\expidx{七}
\expname{各种信号的谱分析}

%%
%% Document body
%%

\begin{document}
% First page
% Some titles and personal information are defined in ``\maketitle''.
\maketitle

\section{实验预习}
1. 线性坐标与对数坐标转换的公式是什么?-6dB电压增益等于多少比例?功率衰减3dB等于衰减多少比例?

2. 将正弦波乘以方波生成一个半波整流的正弦波,计算其频谱。

\section{实验记录与思考题}
\begin{figure}[H]
    \caption{脉冲序列的时域与频域}
    \centering

\end{figure}

观察脉冲序列,其宽度为:\underline{\ \ \ }ms和重复周期为:\underline{\ \ \ }ms。

\textbf{问题 1}

脉冲序列在哪些频率处增益为零?

\textbf{问题 2}

脉冲序列的零增益点之间的频率间隔与脉冲宽度之间的数学关系是什么?

\textbf{问题 3}

$\dfrac{\sin(x)}{x}$ 有什么特性(过零周期、幅度)?

\textbf{问题 4}

当占空比趋近于0时,你发现有怎样的一般趋势?

\textbf{问题 5}

根据上述趋势,预期单脉冲的频谱(即脉冲之间的间隔非常大的脉冲序列)会具有什么形状?

\textbf{问题 6}

假设所观察的Sinc脉冲是关于零点对称的,那么信号在哪些时刻发生零交叉?

\textbf{问题 7}

上述实验中,调节TLPF中心频谱fc和增益Gain时,频域图分别对应哪些变化?

\begin{figure}[H]
    \caption{限幅器四中设置情况下的输出图(包含时域和频域)}
    \centering

\end{figure}

\textbf{问题 8}

限幅对信号的频谱有什么影响?

\textbf{问题 9}

输入频率与输出谐波频率之间的关系是什么?

\textbf{问题 10}

整流正弦波的频谱有哪些特性?与限幅器相比有哪些不同?

\textbf{问题 11}

限幅过程、整流的过程分别是线性还是非线性过程?为什么?

\section{实验过程与数据分析}
 {\kaishu (可以写实验中遇到的问题及解决方式,以及叙述具体实验过程,记录实验数据在原始数据表格,如需要引用原始数据表格,请标注出表头,如“实验记录见表2-*”)}

\section{实验体会与建议}

\end{document}
