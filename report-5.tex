\documentclass{sasreport}

%%
%% User settings
%%

\classno{}
\stuno{}
\groupno{}
\stuname{}
\expdate{\expdatefmt\today}
\expidx{五}
\expname{复数与傅里叶级数}

%%
%% Document body
%%

\begin{document}
% First page
% Some titles and personal information are defined in ``\maketitle''.
\maketitle

\section{实验预习}
 (1) 证明欧拉公式 $e^{i\theta}=\cos(\theta)+i\sin(\theta)$.

(2) 傅里叶级数展开的条件是什么?

(3) 写出周期 $T$、占空比为 $\tau$、幅值为 1 的双极性方波展开的傅里叶级数。

\section{实验记录与思考题}
\textbf{问题 1:}

\textbf{画出 XY 图波形图},并说明XY图呈圆形的原因。

\textbf{问题 2:}

以 $A\cos(\omega t+\theta)$ 函数的形式写出“f+g”的方程表达式。用示波器测量并画出“f+g”的波形,其是否与计算值相符?

\textbf{问题 3:}

保持接线不变,将相位设置为0度和180度,再设置为0度和-180度。这些设置下,求和的输出信号是什么?为什么?

\begin{table}[H]
    \caption{合成幅值读数}
    \centering
    \begin{tabularx}{.8\textwidth}{|c|Y|c|Y|} \hline
        相位 (度) & 合成信号的幅值 ($V_\mathrm{pk}$) & 相位 (度) & 合成信号的幅值 ($V_\mathrm{pk}$) \\\hline
        0      &                           & 210    &                           \\\hline
        30     &                           & 240    &                           \\\hline
        60     &                           & 270    &                           \\\hline
        90     &                           & 300    &                           \\\hline
        120    &                           & 330    &                           \\\hline
        150    &                           & 360    &                           \\\hline
        180    &                           &        &                           \\\hline
    \end{tabularx}
\end{table}

\textbf{问题 4:}

观察表 1 中 $V_\mathrm{pk}$ 的规律,结合欧拉公式,推导合成信号“f+g”的方程。

\textbf{问题 5:}

余弦波的十个谐波都设置为1后,它的峰值振幅是多少?基波是奇函数还是偶函数?合成波是奇函数还是偶函数?

\textbf{问题 6:}

正弦波的十个谐波都设置为1后,它的峰值振幅是多少?基波是奇函数还是偶函数?合成波是奇函数还是偶函数?

\textbf{问题 7:}

上述两个观察到的平均值是多少?你是如何测量的?

\textbf{问题 8:}

上述两个观察到的平均值是多少?你是如何测量的?

\textbf{问题 9:}

在乘法器系统功能定义的基础上,推导问题7或问题8中任意一次平均值的计算过程。

\begin{table}[H]
    \caption{傅里叶级数数据表}
    \centering
    \begin{tabularx}{.8\textwidth}{|c|Y|Y|} \hline
        \makecell{sine harmonic \\ (谐波次数)} & \makecell{直流幅值 \\ (输入正弦波) (V)} & \makecell{直流幅值 \\ (输入余弦波) (V)} \\\hline
        1 &  &                  \\\hline
        2 &  &                  \\\hline
        3 &  &                  \\\hline
        4 &  &                  \\\hline
        5 &  &                  \\\hline
        6 &  &                  \\\hline
        7 &  &                  \\\hline
    \end{tabularx}
\end{table}

\textbf{问题 10:}

表 2 中测量值与计算值是否相符,举例说明。

\begin{table}[H]
    \caption{手动扫频仪数据表格}
    \centering
    \begin{tabularx}{\textwidth}{|c|Y|Y|c|Y|} \hline
        \makecell{输入频率        \\ (Hz)} & \makecell{可调低通滤波器 \\ 峰峰值 ($V_\mathrm{pp}$)} & \makecell{直流分量 \\ 即峰峰值 \\ 的一半 (V)} & \makecell{输入值 \\ (谐波求和器 \\ cos; sin)} & \makecell{理论计算结果 \\ (V)} \\\hline
        1000 &  &  & 1, 0   & \\\hline
        2000 &  &  & 0, 0.3 & \\\hline
        3000 &  &  & 0.5, 1 & \\\hline
        4000 &  &  & 0, 0   & \\\hline
        5000 &  &  & 0, 0   & \\\hline
        6000 &  &  & 1, 0   & \\\hline
        7000 &  &  & 0, 2   & \\\hline
        直流输入 &  &  &        & \\\hline
    \end{tabularx}
\end{table}

\textbf{问题 11:}

举例说明(3000Hz)直流分量的计算公式。

\begin{table}[H]
    \caption{方波信号测量数据表 (选做)}
    \centering
    \begin{tabularx}{.8\textwidth}{|c|Y|Y|Y|} \hline
        输入频率 (Hz) & 直流分量 (V) & 归一化值 (V) & 理论计算 (V) \\\hline
        1000      &          &          &          \\\hline
        2000      &          &          &          \\\hline
        3000      &          &          &          \\\hline
        4000      &          &          &          \\\hline
        5000      &          &          &          \\\hline
        6000      &          &          &          \\\hline
        7000      &          &          &          \\\hline
        直流输入      &          &          &          \\\hline
    \end{tabularx}
\end{table}

\textbf{问题 12 (选做):}

将表 4 的测得的傅里叶级数系数与理论值相比较,是否测量与理论相符合。写出奇次谐波的系数的计算公式?

\textbf{问题 13 (选做):}

解释表 4 中有些谐波分量几乎为0的原因?

\textbf{问题 14 (选做):}

能在占空比为20\%的方波中检测到偶次谐波吗?

\section{实验过程与数据分析}
 {\kaishu (可以写实验中遇到的问题及解决方式,以及叙述具体实验过程,记录实验数据在原始数据表格,如需要引用原始数据表格,请标注出表头,如“实验记录见表2-*”)}

\section{实验体会与建议}

\end{document}
