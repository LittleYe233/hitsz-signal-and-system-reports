\documentclass{sasreport}

%%
%% User settings
%%

\stuno{}
\stuname{}
\expdate{\expdatefmt\today}
\expidx{2}
\expname{MATLAB 基础}

%%
%% This report has a special title. So we modify ``\maketitle'' command.
%%

\fancyhead[C]{}
\renewcommand{\headrulewidth}{0pt}
\setlength{\headheight}{13pt}

\renewcommand{\maketitle}{%
    % Begin the experiment title.
    \noindent{\huge\bfseries{\centering{实验\theexpidx:\theexpname}\par}}

    \begin{table*}[h]
        \setlength\tabcolsep{.5em}
        \begin{tabularx}{\textwidth}{XX}
            {
                \begin{tabularx}{\linewidth}{cY}
                    {\Large 实验日期与时间:} & {\Large \theexpdate} \\\Xcline{2-2}{1pt}
                \end{tabularx}
            } & {
                    \begin{tabularx}{\linewidth}{cY}
                        {\Large 评分:} & \\\Xcline{2-2}{1pt}
                    \end{tabularx}
            }     \\
            % Add to the vertical gap between the two lines.
            % We do not specify `/arraystretch' because it widens the vertical
            % height of cells so that the horizontal lines (`\cline') are too away
            % from the text.
            \\ [-1.2em]
            {
            \begin{tabularx}{\linewidth}{cY}
                {\Large 学生姓名:} & {\Large \thestuname} \\\Xcline{2-2}{1pt}
            \end{tabularx}
            } & {
                    \begin{tabularx}{\linewidth}{cY}
                        {\Large 学生学号:} & {\Large \thestuno} \\\Xcline{2-2}{1pt}
                    \end{tabularx}
            }
        \end{tabularx}
    \end{table*}
}

\ctexset{section={
      format=\Large\bfseries\raggedright,
      number=\chinese{section},
      name={,、},
      aftername=
     }}

%%
%% Document body
%%

\begin{document}
% First page
% Some titles and personal information are defined in ``\maketitle''.
\maketitle

{\large \noindent\textbf{源文件}请按照以下顺序放到一个文件夹内,并将文件夹命名为:\textbf{学号-姓名-实验*},如:123456-张三-实验2,
    \begin{enumerate}
        \item 电子版的实验报告(学号-姓名-实验*.pdf);
        \item 程序源文件:*.m。
    \end{enumerate}
    这个文件夹打包作为实验报告整体提交。}

\section{实验目的}

 {\large \noindent(1)	对MATLAB软件有一个基本的认识;\\
  (2)	理解矩阵(数组)概念及其各种运算和操作;\\
  (3)	掌握绘图函数;\\
  (4)	学会M文件的基本操作。}

\section{实验内容}
 {\large \noindent 要求:实验过分别写到每个实验内容下,需要时给出截图(不要为整个屏幕的截图),最后给出实现这些内容的程序文件(所有内容放到一个M文件中,内容(4)不需要提交M文件)。

  \noindent (1)创建一个任意10*20的随机数组A,A1为数组A中第3行5列到第8行12列组成的子数组,求A1每一列的和、均值、方差。\\
  (2)创建B、C为任意实数组成的3×3数组,分别进行以下计算:\\
  \phantom{(2)}k*B,其中k为任意自定义的实数\\
  \phantom{(2)}B矩阵的3次方\\
  \phantom{(2)}B的每个元素除以C对应的那个元素\\
  \phantom{(2)}B+1j*C\\
  \phantom{(2)}B+1j*C的转置\\
  (3)绘制一条 $\sin(x)$ 曲线,$x$ 的范围在0到4pi之间,要求绘图有名字、横纵坐标有图例,绘图有网格。\\
  (4)将给定的.m文件,调试运行成功,并给出绘图结果。\\
  (5)附加题:\\
  计算第(1)题中数组A的第4行4列到第8行8列的子数组A1和第5行5列到第9行9列的子数组A2对应的每一列的相关系数(自己查公式),要求:当A2中有元素值小于0.5时,该值不参与计算;可以使用if和for语句来操作(语法自己查,与C语言类似),给出源代码和执行结果的截图。}

\section{实验思考题}
 {\large\noindent 1、linspace(1,5,10)表示的意义是什么?}

 {\large\noindent 2、怎么对矩阵的行进行计算(比如分别计算每一行的和、均值、方差)?}

 {\large\noindent 3、如何将对数组进行排序(从大到小、从小到大)?}

\section{实验体会与建议}

\end{document}
