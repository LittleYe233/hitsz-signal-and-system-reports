\documentclass{sasreport}

%%
%% User settings
%%

\classno{}
\stuno{}
\groupno{}
\stuname{}
\expdate{\expdatefmt\today}
\expidx{八}
\expname{拉普拉斯变换}

%%
%% Document body
%%

\begin{document}
% First page
% Some titles and personal information are defined in ``\maketitle''.
\maketitle

\section{实验预习}
1、对于图8- 1中的系统,写出输出 $x_0(t)$ 关于输入 $u(t)$ 的微分方程,系统函数,零极点以及幅频响应和相频响应的表达式。然后将输出 $x_0 = ej\omega t$ 代入该方程,得到所对应的输入 $u(t)$。然后写出输出 $x_1(t)$ 关于输入 $u(t)$ 的微分方程,系统函数以及零极点。

2、对于图8- 2中的系统,写出输出 $y$ 关于输入 $u(t)$ 的微分方程,系统函数,零极点以及幅频响应和相频响应的表达式。

\section{实验记录与思考题}
\begin{figure}[H]
    \caption{实验实测以 $x_0$ 为输出的幅频响应曲线}
    \centering

\end{figure}

\textbf{问题 1}

由你已经写出的系统函数画出零极点图和大致的幅频曲线图,判断这是一个什么性质滤波器?

\begin{figure}[H]
    \caption{理论计算以 $x_0$ 为输出的零极点图以及幅频响应曲线}
    \centering

\end{figure}

\textbf{问题 2}

3dB截止频率是多少?增益最大处对应频率是多少?

\textbf{问题 3}

为了与以 $x_0$ 点为输出的响应进行比较,我们在以 $x_1$ 节点为输出处测量频率响应。这是一个什么性质滤波器?最大增益处的频率及增益分别是多少?3db截止频率是多少?3db带宽是多少?记录并画出幅频响应曲线于图 3 中。

\begin{figure}[H]
    \caption{实验实测以 $x_1$ 为输出的幅频响应曲线}
    \centering

\end{figure}

\begin{figure}[H]
    \caption{输入与 $x_0$ 输出的时域图状}
    \centering

\end{figure}

\textbf{问题 4}

在拉普拉斯变换中,系统稳定的条件是什么?

\textbf{问题 5}

逐渐调高 $a_1$ 的值到正数,观察 $x_0$ 输出变化情况。1)描述输出变化情况;2)当 $a_1=0.01$ 时,发生了什么现象?记录此时输出的波形于图 5 中(此时你可以切断输入观察现象)。

\begin{figure}[H]
    \caption{$x_0$ 输出时域图}
    \centering

\end{figure}

\textbf{问题 6}

由你已经写出的系统函数画出零极点图和大致的幅频曲线图,判断该幅频响应与哪种滤波器幅频响应类似?

\begin{figure}[H]
    \caption{理论计算以 $y$ 为输出的零极点图以及幅频响应曲线}
    \centering

\end{figure}

\textbf{问题 7}

测量并画出以 $y$ 为输出的频率响应。并记录在图 7 中。增益最小处的频率是多少?

\begin{figure}[H]
    \caption{实验实测以 $y$ 为输出的幅频响应曲线}
    \centering

\end{figure}

\textbf{问题 8 (选做)}

怎么调节 $b_2,b_1,b_0$ 参数可以使系统变为全通滤波器?

\section{实验过程与数据分析}
 {\kaishu (可以写实验中遇到的问题及解决方式,以及叙述具体实验过程,记录实验数据在原始数据表格,如需要引用原始数据表格,请标注出表头,如“实验记录见表2-*”)}

\section{实验体会与建议}

\end{document}
