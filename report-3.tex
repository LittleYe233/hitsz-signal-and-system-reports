\documentclass{sasreport}

%%
%% User settings
%%

\classno{}
\stuno{}
\groupno{}
\stuname{}
\expdate{\expdatefmt\today}
\expidx{三}
\expname{线性与非线性系统}

%%
%% Document body
%%

\begin{document}
% First page
% Some titles and personal information are defined in ``\maketitle''.
\maketitle

\section{实验前思考题}
 (1) 写出用二倍角作为自变量来表达正弦波平方的公式:

(2) 线性的定义是什么:

\section{实验记录}
\begin{table}[H]
    \caption{限幅器实验结果}
    \centering\large
    \begin{tabularx}{.8\textwidth}{|Y|Y|Y|}
        \hline
        输入幅值 ($V_\mathrm{pp}$) & 限幅器幅值 ($V_\mathrm{pp}$) & 整流器幅值 ($V_\mathrm{pp}$) \\\hline
        1                      &                         &                         \\\hline
        2                      &                         &                         \\\hline
        3                      &                         &                         \\\hline
        4                      &                         &                         \\\hline
        5                      &                         &                         \\\hline
        6                      &                         &                         \\\hline
    \end{tabularx}
\end{table}

\begin{table}[H]
    \caption{乘法器实验结果}
    \centering\large
    \begin{tabularx}{.6\textwidth}{|Y|Y|}
        \hline
        输入幅值 ($V_\mathrm{pp}$) & 乘法器幅值 ($V_\mathrm{pp}$) \\\hline
        1                      &                         \\\hline
        2                      &                         \\\hline
        3                      &                         \\\hline
        4                      &                         \\\hline
        5                      &                         \\\hline
        6                      &                         \\\hline
    \end{tabularx}
\end{table}

\begin{table}[H]
    \caption{VCO 系统输出}
    \centering\large
    \begin{tabularx}{.6\textwidth}{|Y|Y|}
        \hline
        DC 输入电压 ($V$) & VCO 系统输出频率 ($Hz$) \\\hline
        -3            &                   \\\hline
        -2            &                   \\\hline
        -1            &                   \\\hline
        0             &                   \\\hline
        1             &                   \\\hline
        2             &                   \\\hline
        3             &                   \\\hline
    \end{tabularx}
\end{table}

\begin{figure}[H]
    \caption{积分器处理前后的信号}
    \centering

\end{figure}

\begin{figure}[H]
    \caption{$S(x_1+x_2)$ 信号与 $Sx_1+Sx_2$ 信号}
    \centering

\end{figure}

\section{实验思考题}
\textbf{问题 1:}

将CH0通道连接至FUNC OUT,CH1通道连接至限幅器的输出端,观察输出信号并回答限幅器满足线性吗?如果不是,请说明理由;如果是,请说明从哪个输入幅值开始,其斜率大概为多少?

\textbf{问题 2:}

将CH0通道连接至FUNC OUT,CH1通道连接至整流器的输出端,观察输出信号,整流器满足线性测试吗?如果不是,请说明理由;如果是,请说明从哪个输入幅值开始,其斜率大概为多少?

\textbf{问题 3:}

将CH0通道连接至FUNC OUT,CH1通道连接至乘法器的输出端,观察输出信号,描述输入和输出之前的关系。

\textbf{问题 4:}

将CH0通道连接至DAC-1输出,CH1通道连接至FUNC OUT,观察输出信号,描述输入电压和输出频率之间的关系。

\textbf{问题 5:}

根据本次实验,结果表明积分器符合线性条件吗?改变一个系统的增益,观察结果,改变增益还符合线性条件吗?(确保增益 $a_1=b_1$、$a_0=b_0$ 且 $a_2=b_2=0$)

\section{实验过程与数据分析}
 {\kaishu (可以写实验中遇到的问题及解决方式,以及叙述具体实验过程,记录实验数据在原始数据表格,如需要引用原始数据表格,请标注出表头,如“实验记录见表2-*”)}

\section{实验体会与建议}

\end{document}
