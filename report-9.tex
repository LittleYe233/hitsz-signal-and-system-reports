\documentclass{sasreport}

%%
%% User settings
%%

\classno{}
\stuno{}
\groupno{}
\stuname{}
\expdate{\expdatefmt\today}
\expidx{九}
\expname{RC网络的时域分析}

%%
%% Document body
%%

\begin{document}
% First page
% Some titles and personal information are defined in ``\maketitle''.
\maketitle

\section{实验预习}
1. 根据阶跃响应的实验原理回答:

(1) 绘制式9-5的示意图,其中,a=1000;

(2) 随着t无限增加,阶跃响应的渐进值如何变化?

2. 脉冲函数的主要性质是什么?

3. 根据卷积的应用的实验原理回答:

(1) 在 $t>0$ 的范围内,绘制式9-8的示意图;

(2) 当 $a_1=a_2$ 时,式9-8可以怎么简化?

4. 根据正弦波输入的响应的实验原理回答:

当 $1/RC = 1000$ (rad/sec) 时,绘制式9-12的示意图,并找出比值为3dB的 $\omega$ 值。

\section{实验记录与思考题}
\begin{figure}[H]
    \caption{RC网络的阶跃响应图}
    \centering

\end{figure}

\textbf{问题 1}

通过示波器测量,RC网络的时间常数为\underline{\ \ \ \ \ }ms,阶跃响应的幅值为\underline{\ \ \ \ \ }V,需要多长时间上升至比最高电平低37\%的水平?并在图1中作出相应的标记。

\textbf{问题 2}

根据实验原理,计算预期阶跃响应信号,其中R = 10,000 ohm,C = 100nF;因此,RC = $1\times 10^{-3}$,即1/RC = $10^3$ = 1000。与实际测量是否相符?

\begin{figure}[H]
    \caption{RC网络的冲激响应图(5\%@50Hz,1\%@100Hz)}

\end{figure}

\textbf{问题 3}

5\%@50Hz,1\%@100Hz脉冲宽度分别是多少?最大幅值分别是多少?

\textbf{问题 4}

由于输入信号为窄脉冲而非冲激信号,输出信号的幅值无法达到理论值。根据实验原理,测量RC回路的时间常数,并判断是否与理论相符?

\begin{figure}[H]
    \caption{指数脉冲响应图}

\end{figure}

\textbf{问题 5}

根据式9-8,推导指数输入脉冲激励下RC网络的输出波形的表达式,并用MATLAB绘制该曲线。测量结果与此网络的理论输出是否相符?

\begin{figure}[H]
    \caption{合成系统的阶跃和冲激响应}

\end{figure}

\textbf{问题 6}

当 $a_0$ 和 $a_1$ 的值是多少的时候,合成系统与实际RC网络最相似?输出信号的方程是什么?与此网络的理论输出是否相符?

\section{实验过程与数据分析}
 {\kaishu (可以写实验中遇到的问题及解决方式,以及叙述具体实验过程,记录实验数据在原始数据表格,如需要引用原始数据表格,请标注出表头,如“实验记录见表2-*”)}

\section{实验体会与建议}

\end{document}
